\section{\color{Red}Appendix A}


%Table \ref{tab:AideMemoire} has some reminders of basic notation. 


%\begin{table}
%	\centering
\center{
		\begin{tabular}{|c|l|}
		  \hline
			
			\textbf{Notation} & \textbf{Description} \\ 
			\hline \hline
			
			$x\in C$ & $x$ is an \textbf{element} or
			  \textbf{element} of collection $C$ \\ \hline
			
			$\forall x \in C\ldots$ & For every element
			  $x$ of collection $C$ \ldots \\ \hline
			
			$\exists x \in C\ldots$ & There exists an 
			  element $x$ in collection $C$\ldots\\ \hline
			
			$x\oplus y$ & \parbox{4.5in}
			  {\headroom Combining $x$ and $y$ with a generalized 
			    operator, $\oplus$, similar to the familiar 
			    $x+y$ notation}\\ \hline
			
			$f:X\rightarrow Y$ & \parbox{4.5in}{\headroom The 
			  function $f$ takes or maps elements of the set
			  $X$ to elements of the set $Y$\buttroom}\\ \hline
			
			$f(x)=E$ & \parbox{4.5in}{\headroom The function $f$, 
			  given the input value $x$, produces the output
			  value $E$ \buttroom} \\ \hline
			    
			$\lambda x.E$ & \parbox{4.5in}{\headroom This entire
			  expression denotes the anonymous
			  function that takes an $x$ and `returns' $E$      
			  \buttroom}\\ \hline
			
			$x : \T$ & $x$ is of type $\T$, or \emph{has} 
			  type $\T$ (\textbf{type assertion})\\ \hline
			
			$\Gamma \vdash x : \T$ & \parbox{4.5in}
			  {\headroom In the context of $\Gamma$, which is
			  a comma-delimited sequence of type assertions, 
			  $x$ has type $\T$}\\ \hline
			
			$P\lor Q, P\land Q$ & $P$ or $Q$ (inclusive or); 
			  $P$ and $Q$\\ \hline
			
			$\lnot P$ & Not $P$; $P$ is false (logical negation;
			  very tight precedence binding)\\ \hline
			
			\rule{0ex}{3.2ex}$\dfrac{P}{Q}$ or $P\Rightarrow Q$ 
			  & If $P$ is true, then $Q$ is true; same as $\lnot 
			  Q \lor P$ (\textbf{implication})\\ \hline
			
			$\{a,b,\ldots\}$ & A \textbf{set} of elements:
			  unordered; duplicates not allowed \\ \hline
			
			$[a, b, \ldots]$ & A \textbf{sequence} or
			  \textbf{list} of elements: ordered; duplicates
			  allowed\\ \hline
			    
			$[a,b]$ & An \textbf{ordered pair}; special case 
			  of list\\ \hline
			
			$\LBag a, b,\ldots\RBag$ & A \textbf{bag} or
			  \textbf{multiset} of elements: unordered; 
			  duplicates allowed\\ \hline
			
			$\LPerm a, b, \ldots\RPerm$ & A \textbf{permutation} 
			  of elements: ordered; duplicates not 
			  allowed \\ \hline
			
			$\{\T\}, [\T], \LBag \T\RBag, \LPerm \T\RPerm$ 
			  & The type of (sets, lists, bags, permutations)
			  of instances of type $\T$ \\ \hline
			  
		\end{tabular}
			\label{tbl:aide-memoire}
}

